%%=============================================================================
%% Inleiding
%%=============================================================================

\chapter{\IfLanguageName{dutch}{Inleiding}{Introduction}}%
\label{ch:inleiding}
In een snel evoluerende wereld van softwareontwikkeling is het cruciaal om
zorgvuldig de meest geschikte technologieën te selecteren bij het ontwikkelen
van een applicatie of tool die het best voldoet aan de specifieke behoeften van
de klant. Het kiezen van de juiste technologieën is essentieel, aangezien ze de
ruggengraat van het eindresultaat vormen en bepalen het succes ervan. Dit is
van uiterst belang om zo te kunnen voldoen aan de specifieke bijnodigdheden en
behoeften van de klant, waardoor tijdsverlies en onnodige kosten worden
vermeden.
\section{\IfLanguageName{dutch}{Probleemstelling}{Problem Statement}}%
\label{sec:probleemstelling}
Dit onderzoek richt zich op de ontwikkeling van de managementtool `Hank', met
als doel een diepgaande analyse te maken van de verschillende front- en
backend-technologieën. Hank: een Elmos-inhouse visualisatie-tool die de
Competence Centers in een live dashboard weergeeft. Elke collega/consultant
heeft een groeipad. Hank toont hiervan de live leertrajecten,
cer\-ti\-fi\-caten maar ook ex\-amens en geplande opleidingen. Deze is een
webapplicatie die in opdracht van het consultancy bedrijf Elmos NV ontwikkeld
zal worden. Hetgeen dat Elmos wil bereiken is het versnellen van
bedrijfsprocessen door middel van een op maat gemaakte managementapplicatie.
Deze webapplicatie zou probleemloos hand in hand gaan met een efficiënte front-
en backend technologie. \bigbreak Er wordt gestreefd naar een robuuste,
performante, schaalbare, redundante, ease of use en kostenefficiënte oplossing
te bieden voor de managementtool `Hank`. Om deze doelstellingen te kunnen
bereiken is het van uiterst belang om de juiste technologieën te selecteren.
Tijdens het onderzoek zal er ook een proof of concept uitgewerkt worden met de
vastgestelde technologieën.

\section{\IfLanguageName{dutch}{Onderzoeksvraag}{Research question}}%
\label{sec:onderzoeksvraag}
% TODO
% Wees zo concreet mogelijk bij het formuleren van je onderzoeksvraag. Een
% onderzoeksvraag is trouwens iets waar nog niemand op dit moment een antwoord
% heeft (voor zover je kan nagaan). Het opzoeken van bestaande informatie (bv.
% ``welke tools bestaan er voor deze toepassing?'') is dus geen onderzoeksvraag.
% Je kan de onderzoeksvraag verder specifiëren in deelvragen. Bv.~als je
% onderzoek gaat over performantiemetingen, dan

Welke front- en backend-technologieën zijn het meest geschikt op het vlak van prestaties, veiligheid, schaalbaarheid voor de ontwikkeling van de managementtool `Hank`?
Welke tools zullen we gebruiken bij het ontwikkelen van de managementtool `Hank`?

\section{\IfLanguageName{dutch}{Onderzoeksdoelstelling}{Research objective}}%
\label{sec:onderzoeksdoelstelling}

% Wat is het beoogde resultaat van je bachelorproef? Wat zijn de criteria voor succes? Beschrijf die zo concreet mogelijk. Gaat het bv.\ om een proof-of-concept, een prototype, een verslag met aanbevelingen, een vergelijkende studie, enz.

Het beoogde resultaat van deze bachelorproef is een Elmos inhouse
VISualisatie-tool: HANK. De VISualisatie-tool die onze Competence Centers in
een live dashboard stopt. HANK houdt leertrajecten, certificeringen, examens en
geplande opleidingen bij voor al onze collega's/consultants.
\section{\IfLanguageName{dutch}{Opzet van deze bachelorproef}{Structure of this bachelor thesis}}%
\label{sec:opzet-bachelorproef}

% Het is gebruikelijk aan het einde van de inleiding een overzicht te
% geven van de opbouw van de rest van de tekst. Deze sectie bevat al een aanzet
% die je kan aanvullen/aanpassen in functie van je eigen tekst.

De rest van deze bachelorproef is als volgt opgebouwd:

In Hoofdstuk~\ref{ch:stand-van-zaken} wordt een overzicht gegeven van de stand
van zaken binnen het onderzoeksdomein, op basis van een literatuurstudie.

In Hoofdstuk~\ref{ch:methodologie} wordt de methodologie toegelicht en worden
de gebruikte onderzoekstechnieken besproken om een antwoord te kunnen
formuleren op de onderzoeksvragen.

% TODO: Vul hier aan voor je eigen hoofstukken, één of twee zinnen per hoofdstuk

In Hoofdstuk~\ref{ch:conclusie}, tenslotte, wordt de conclusie gegeven en een
antwoord geformuleerd op de onderzoeksvragen. Daarbij wordt ook een aanzet
gegeven voor toekomstig onderzoek binnen dit domein.