%==============================================================================
% Sjabloon onderzoeksvoorstel bachproef
%==============================================================================
% Gebaseerd op document class `hogent-article'
% zie <https://github.com/HoGentTIN/latex-hogent-article>

% Voor een voorstel in het Engels: voeg de documentclass-optie [english] toe.
% Let op: kan enkel na toestemming van de bachelorproefcoördinator!
\documentclass{hogent-article}

% Invoegen bibliografiebestand
\addbibresource{voorstel.bib}

% Informatie over de opleiding, het vak en soort opdracht
\studyprogramme{Professionele bachelor toegepaste informatica}
\course{Bachelorproef}
\assignmenttype{Onderzoeksvoorstel}
% Voor een voorstel in het Engels, haal de volgende 3 regels uit commentaar
% \studyprogramme{Bachelor of applied information technology}
% \course{Bachelor thesis}
% \assignmenttype{Research proposal}

\academicyear{2023-2024} % TODO: pas het academiejaar aan

% TODO: Werktitel
\title{Optimalisatie van Frontend- en Backend-Technologieën voor de Ontwikkeling van de Managementtool "Hank"}

% TODO: Studentnaam en emailadres invullen
\author{Alexander Schatteman}
\email{Alexander.schatteman@student.hogent.be}

% TODO: Geef de co-promotor op
\supervisor[Co-promotor]{K. DeRudder (Elmos NV, \href{mailto:kevin.derudder@elmos.be}{kevin.derudder@elmos.be})}

% Binnen welke specialisatierichting uit 3TI situeert dit onderzoek zich?
% Kies uit deze lijst:
%
% - Mobile \& Enterprise development
% - AI \& Data Engineering
% - Functional \& Business Analysis
% - System \& Network Administrator
% - Mainframe Expert
% - Als het onderzoek niet past binnen een van deze domeinen specifieer je deze
%   zelf
%
\specialisation{Mobile \& Enterprise development}
\keywords{Frontend, Backend, $\lambda$-calculus}

\begin{document}
In deze paper zal er een gedetailleerde vergelijking worden uitgevoerd doormiddel van benchmarks om de snelheid, betrouwbaarheid, schaalbaarheid, 
energie-efficientie en kostprijs van verschillende frontend en backend Technologieën te beoordelen.
Deze data zal gebruikt worden bij het ontwikkelen van de managementtool "Hank" om 
zo een meest geschikte combinatie van technologieën te identificeren. 
Als er gesproken wordt over frontend technologieën bedoelen we de technologie die gebruikt wordt 
om de gebruiker een interface te geven waarop de gebruiker de verschillende bedrijfsprocessen kan beheren. 
Deze interface communiceert met de backend technologie die de data verwerkt en opslaat.
Als we over backend programmeertalen spreken hebben we een aantal populaire opties zoals Java, Python, C\#, PHP, Javascript, ...
Hierop zijn er een aantal frameworks gebaseerd zoals Spring, Django, .NET, Laravel, Node.js, ...
Deze frameworks zijn uniek in hun eigen manier en hebben elk hun eigen voor- en nadelen. 
Bij frontend technologieën hebben we ook een aantal populaire opties zoals Javascript, Typescript, HTML, CSS, ...
Hierop zijn respectievelijk ook een aantal frameworks gebaseerd zoals Angular, React, Vue, ...
We hopen met dit onderzoek een beter inzicht te krijgen in de verschillende technologieën en zo 
een kostefficiente en performante oplossing te kunnen bieden voor de managementtool "Hank".
\begin{abstract}


\end{abstract}

\tableofcontents

% De hoofdtekst van het voorstel zit in een apart bestand, zodat het makkelijk
% kan opgenomen worden in de bijlagen van de bachelorproef zelf.
\input{voorstel-inhoud}

\printbibliography[heading=bibintoc]

\end{document}