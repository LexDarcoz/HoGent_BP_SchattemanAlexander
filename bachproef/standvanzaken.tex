\chapter{\IfLanguageName{dutch}{Stand van zaken}{State of the art}}%
\label{ch:stand-van-zaken}

% Tip: Begin elk hoofdstuk met een paragraaf inleiding die beschrijft hoe
% dit hoofdstuk past binnen het geheel van de bachelorproef. Geef in het
% bijzonder aan wat de link is met het vorige en volgende hoofdstuk.

% Pas na deze inleidende paragraaf komt de eerste sectiehoofding.

% Dit hoofdstuk bevat je literatuurstudie. De inhoud gaat verder op de inleiding, maar zal het onderwerp van de bachelorproef *diepgaand* uitspitten. De bedoeling is dat de lezer na lezing van dit hoofdstuk helemaal op de hoogte is van de huidige stand van zaken (state-of-the-art) in het onderzoeksdomein. Iemand die niet vertrouwd is met het onderwerp, weet nu voldoende om de rest van het verhaal te kunnen volgen, zonder dat die er nog andere informatie moet over opzoeken \autocite{Pollefliet2011}.

% Je verwijst bij elke bewering die je doet, vakterm die je introduceert, enz.\ naar je bronnen. In \LaTeX{} kan dat met het commando \texttt{$\backslash${textcite\{\}}} of \texttt{$\backslash${autocite\{\}}}. Als argument van het commando geef je de ``sleutel'' van een ``record'' in een bibliografische databank in het Bib\LaTeX{}-formaat (een tekstbestand). Als je expliciet naar de auteur verwijst in de zin (narratieve referentie), gebruik je \texttt{$\backslash${}textcite\{\}}. Soms is de auteursnaam niet expliciet een onderdeel van de zin, dan gebruik je \texttt{$\backslash${}autocite\{\}} (referentie tussen haakjes). Dit gebruik je bv.~bij een citaat, of om in het bijschrift van een overgenomen afbeelding, broncode, tabel, enz. te verwijzen naar de bron. In de volgende paragraaf een voorbeeld van elk.

% \textcite{Knuth1998} schreef een van de standaardwerken over sorteer- en zoekalgoritmen. Experten zijn het erover eens dat cloud computing een interessante opportuniteit vormen, zowel voor gebruikers als voor dienstverleners op vlak van informatietechnologie~\autocite{Creeger2009}.

% Let er ook op: het \texttt{cite}-commando voor de punt, dus binnen de zin. Je verwijst meteen naar een bron in de eerste zin die erop gebaseerd is, dus niet pas op het einde van een paragraaf.

Een framework is een set van tools gebruikt als basis om goed gestructureerde
en betrouwbare software en systemen te ontwikkelen. Een web development
framework wordt vaak gebruikt bij het ontwikkelen van een web applicatie.
Inclusief web services, web resources en web API's.

\subsection{Front-end frameworks}%
\label{sub:frontend_frameworks}
Front-end van een applicatie, vaak duidt dit naar de programmeertalen HTML, CSS en Ja\-va\-script. Deze programmeertalen vormen de basis van een webapplicatie. HTML zorgt voor de structuur van de webpagina, CSS zorgt voor de stijl en Ja\-va\-script zorgt voor de interactie. Maar deze zijn niet voldoende om een volledige webapplicatie te ontwikkelen. Daarom zijn er een aantal frameworks ontwikkeld die de ontwikkeling van een webapplicatie vergemakkelijken.
Voorbeelden zijn Angular, React, Ambular, Ember, Bootstrap, Vue, \ldots\autocite{Jaiswal2022}

\subsection{Back-end frameworks}%
\label{sub:backend_frameworks}
Back-end van een applicatie, vaak duidt dit naar de programmeertalen Java, Python, C\#, PHP, Ja\-va\-script\ldots  Deze programmeertalen zorgen voor de logica van de applicatie. Maar deze zijn niet voldoende om een volledige webapplicatie te ontwikkelen. Ook hier zijn er een aantal frameworks ontwikkeld die de ontwikkeling van een webapplicatie vergemakkelijken.
Voorbeelden zijn Spring, Django,.NET, Node.js, \ldots\autocite{Kaluza2019}

\subsection{Push-Based vs. Pull-Based}%
Een belangrijk onderscheid in web development architecturen is het verschil
tussen ``Push-Based'' en ``Pull-Based'' architecturen, dat verwijst naar de rol
van de server in relatie tot de view layer.\autocite{Lomas2022} \bigbreak In
``Push-Based'' architecture start de server met het verzenden (Pushen) van
gegevens om vervolgens de resultaten op de view te renderen. Veel MVC
(Model-view-controller) frameworks zijn gebaseerd op dit type architectuur.
Voorbeelden van push-based technologieën zijn; Django, Ruby on Rails, Symfony,
Spring MVC, Stripes en codeIgniter. \bigbreak Alternatief, bestaan ook
``Pull-Based'' architecturen, deze haalt de view laag resultaten op van
meerdere controllers volgens de vereisten. Voorbeelden van deze architecturen
zijn Lift, Tapestry, JBoss Seam en Wicket. \bigbreak Dit onderscheid is
essentieel bij het kiezen van een geschikt framework, omdat het de manier
waarop de gegevens worden uitgewisseld beïnvloedt.

\subsection{Ge\-rel\-at\-eerde studies}%
\label{sub:Gerelateerde_studies}
Er zijn reeds verschillende studies uitgevoerd in verband met de pre\-sta\-ties en schaalbaarheid van frameworks\autocite{Daityari2023}.  Deze onderzoeken omvatten ook analyses van de beschikbaarheid van soft\-ware\-ont\-wik\-kel\-aars die vertrouwd zijn met deze frameworks.

\subsection{React vs Vue vs Angular}%
Angular is ontwikkeld door Google, deze onderscheidt zich als een volledig
framework dat is gebaseerd op TypeScript. In vergelijking met React, een
populair front-end framework ontwikkeld door Facebook, dat zijn basis heeft in
JavaScript en voornamelijk dient als een open-source bibliotheek voor het
uitbouwen van gebruikersinterfaces. Bij het gebruik van React voor
webapplicatieontwikkeling zijn aanvullende packages vereist. Aan de andere kant
is Vue een progressief framework voor het ontwikkelen van gebruikersinterfaces,
gecreëerd door Evan You.\autocite{EvanYou2024} Deze wordt best gebruikt in
minder complexe, kleinere web applicaties en is ook makkelijker om te leren.
Angular daarentegen gaat verder dan alleen een framework; het wordt
gepresenteerd als een uitgebreid platform voor het bouwen van single-page
client-applicaties met behulp van HTML en TypeScript. Angular wordt vaak ook
beschouwd als een framework met een steile leercurve. De keuze tussen Angular,
React en Vue hangt af van de behoeften, budget en complexiteit van het
project.\autocite{Joshi2023}

\subsection{Node vs .NET}%
NodeJS en .NET zijn beide krachtige frameworks met verschillende
karakteristieken. De keuze tussen Node.js en .NET hangt af van de vereisten van
het project. Node.js kan uitblinken in situaties waar snelle prestaties en
schaalbaarheid van cruciaal belang zijn, terwijl .NET de voorkeur kan hebben
bij het bouwen van meer complexe enterprise-toepassingen die moeten samenwerken
met de Microsoft-ecosfeer. \autocite{Hutsulyak2023}

\bigbreak
In de verdere verkenning van ge\-rel\-a\-teer\-de studies is het belangrijk te kijken naar specifieke bevindingen, methodologieën en conclusies die door onderzoekers zijn bereikt. Het identificeren van trends, best practices en e\-ven\-tu\-ele uitdagingen die in eerdere studies naar voren zijn gekomen, kan waardevolle inzichten bieden bij het begrijpen van de rol van frameworks in webontwikkeling en het beoordelen van hun geschiktheid voor specifieke projecten.

\lipsum[7-20]
