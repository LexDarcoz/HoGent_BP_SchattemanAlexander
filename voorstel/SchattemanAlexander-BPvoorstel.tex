%==============================================================================
% Sjabloon onderzoeksvoorstel bachproef
%==============================================================================
% Gebaseerd op document class `hogent-article'
% zie <https://github.com/HoGentTIN/latex-hogent-article>

% Voor een voorstel in het Engels: voeg de documentclass-optie [english] toe.
% Let op: kan enkel na toestemming van de bachelorproefcoördinator!
\documentclass{hogent-article}

% Invoegen bibliografiebestand
\addbibresource{voorstel.bib}

% Informatie over de opleiding, het vak en soort opdracht
\studyprogramme{Professionele bachelor toegepaste informatica}
\course{Bachelorproef}
\assignmenttype{Onderzoeksvoorstel}
% Voor een voorstel in het Engels, haal de volgende 3 regels uit commentaar
% \studyprogramme{Bachelor of applied information technology}
% \course{Bachelor thesis}
% \assignmenttype{Research proposal}

\academicyear{2023-2024} % TODO: pas het academiejaar aan

% TODO: Werktitel
\title{Evaluatie van de Impact van een Op Maat Gemaakte Management Tool Hank, op de Efficiëntie van Bedrijfsprocessen}

% TODO: Studentnaam en emailadres invullen
\author{Alexander Schatteman}
\email{Alexander.schatteman@student.hogent.be}

% TODO: Geef de co-promotor op
\supervisor[Co-promotor]{K. DeRudder (Elmos NV, \href{mailto:kevin.derudder@elmos.be}{kevin.derudder@elmos.be})}

% Binnen welke specialisatierichting uit 3TI situeert dit onderzoek zich?
% Kies uit deze lijst:
%
% - Mobile \& Enterprise development
% - AI \& Data Engineering
% - Functional \& Business Analysis
% - System \& Network Administrator
% - Mainframe Expert
% - Als het onderzoek niet past binnen een van deze domeinen specifieer je deze
%   zelf
%
\specialisation{Mobile \& Enterprise development}
\keywords{Scheme, World Wide Web, $\lambda$-calculus}

\begin{document}

\begin{abstract}
Dit onderzoek richt zich op de evaluatie van de impact van de op maat gemaakte managementtool "Hank" op de efficiëntie van bedrijfsprocessen binnen het consultancybedrijf ELMOS. Hank is specifiek ontworpen om HR-personeelsleden de mogelijkheid te bieden certificaten te koppelen aan alle werknemers in de organisatie.
De studie omvat een diepgaande analyse van de implementatie en het gebruik van Hank binnen ELMOS, met de nadruk op de verbeteringen in het beheer van certificaten en de algehele efficiëntie van HR-processen. Methodologieën zoals enquêtes, interviews en prestatie-indicatoren worden toegepast om zowel kwalitatieve als kwantitatieve gegevens te verzamelen.
De verwachting is dat Hank zal bijdragen aan een gestroomlijnd certificeringsproces, wat op zijn beurt de compliance en competenties van medewerkers zal versterken. De resultaten van dit onderzoek kunnen waardevolle inzichten bieden voor vergelijkbare organisaties die overwegen om op maat gemaakte managementtools te implementeren om hun HR-processen te optimaliseren en de algehele bedrijfsefficiëntie te verhogen.
\end{abstract}

\tableofcontents

% De hoofdtekst van het voorstel zit in een apart bestand, zodat het makkelijk
% kan opgenomen worden in de bijlagen van de bachelorproef zelf.
\input{voorstel-inhoud}

\printbibliography[heading=bibintoc]

\end{document}