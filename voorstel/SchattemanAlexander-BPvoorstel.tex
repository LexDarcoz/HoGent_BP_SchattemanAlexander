%==============================================================================
% Sjabloon onderzoeksvoorstel bachproef
%==============================================================================
% Gebaseerd op document class `hogent-article'
% zie <https://github.com/HoGentTIN/latex-hogent-article>
%
\documentclass{hogent-article}
\usepackage{pgfgantt}
\usepackage{tikz}
% Invoegen bibliografiebestand
\addbibresource{voorstel.bib}

% Informatie over de opleiding, het vak en soort opdracht
\studyprogramme{Professionele bachelor toegepaste informatica}
\course{Bachelorproef}
\assignmenttype{Onderzoeksvoorstel}

\academicyear{2023-2024}

% TODO: Werktitel
\title{Onderzoek naar de meest geschikte technologie stacks voor de ontwikkeling van de managementtool `Hank`}

\author{Alexander Schatteman}
\email{Alexander.schatteman@student.hogent.be}

% TODO: Geef de co-promotor op
\supervisor[Co-promotor]{K. DeRudder (Elmos NV, \href{mailto:kevin.derudder@elmos.be}{kevin.derudder@elmos.be})}

\projectrepo{https://github.com/LexDarcoz/HoGent_BP_SchattemanAlexander}
\specialisation{Mobile \& Enterprise development}
\keywords{Frontend, Backend, Interface, Managementtool, HANK, Frameworks, Kostprijs, Performantie, Betrouwbaarheid, Schaalbaarheid}

% ----------- Begin Paper --------------
\begin{document}

\begin{abstract}
    In een snel e\-vo\-lu\-erende wereld van softwareontwikkeling en innovatie is het cruciaal om de meest geschikte technologieën te se\-lec\-teren bij het ontwikkelen van een applicatie of tool. Zodat deze het best voldoet aan de specifieke behoeften van de klant.
    \bigbreak
    Het hoofddoel van deze paper is het onderzoeken welke combinatie van front- en backend technologieën het meest geschikt is voor de ontwikkeling van de managementtool `Hank`.
    \bigbreak
    Om de context van `HANK' te verduidelijken: `HANK' is een managementtool dat tijdens dit onderzoek ook daadwerkelijk zal ontwikkeld worden. In opdracht van consultancy bedrijf Elmos NV, heeft `HANK' tot doel de bedrijfsprocessen van het Human Resources-departement efficiënter te maken door een gecentraliseerde hub te bieden voor documenten, opleidingen en specialiteiten van consultants die bij het bedrijf werken. Hierdoor zou het selectieproces bij opdrachten soepeler en efficiënter moeten verlopen wanneer het bedrijf zich aan een project wijdt. Deze webapplicatie zou probleemloos hand in hand moeten gaan met een efficiënte front- en backend technologie. 
    \bigbreak 
    De vergelijking zal worden gemaakt op basis van performantie, betrouwbaarheid en schaalbaarheid. In context van dit onderzoek verwijst de term `Frontend-technologieën' naar de technologische componenten die worden ingezet om de gebruiker een intuïtieve interface te bieden voor het beheren van diverse bedrijfsprocessen. Deze interface communiceert met de backend technologie die de data verwerkt en opslaat. 
    \bigbreak
    Dit onderzoek beoogt een beter inzicht te bezorgen in de verschillende technologieën en zo een kostenefficiënte en performante oplossing te kunnen bieden voor de managementtool `Hank'.
\end{abstract}

\tableofcontents
%---------- Inleiding ----------------------
\section{Introductie}%
\label{sec:introductie}
  In een snel evoluerende wereld van softwareontwikkeling is het cruciaal om zorgvuldig de meest geschikte technologieën te selecteren bij het ontwikkelen van een applicatie of tool die het best voldoet aan de specifieke behoeften van de klant.
  Het kiezen van de juiste technologieën is essentieel, aangezien ze de ruggengraat van het eindresultaat vormen en bepalen het succes ervan. Dit is van uiterst belang om zo te kunnen voldoen aan de specifieke bijnodigdheden en behoeften van de klant, waardoor tijdsverlies en onnodige kosten worden vermeden. 
  
  \subsection{Probleemstelling}%
  \label{sub:probleemstelling}
  Dit onderzoek richt zich op de ontwikkeling van de managementtool `Hank', met als doel een diepgaande analyse te maken van de verschillende front- en backend-technologieën. Hank: een Elmos-inhouse visualisatie-tool die de Competence Centers in een live dashboard weergeeft. Elke collega/consultant heeft een groeipad. Hank toont hiervan de live leertrajecten, cer\-ti\-fi\-caten maar ook ex\-amens en geplande opleidingen. Deze is een webapplicatie die in opdracht van het consultancy bedrijf Elmos NV ontwikkeld zal worden. Hetgeen dat Elmos wil bereiken is het versnellen van bedrijfsprocessen door middel van een op maat gemaakte managementapplicatie. Deze webapplicatie zou probleemloos hand in hand gaan met een efficiënte front- en backend technologie.
  \bigbreak
  Er wordt gestreefd naar een robuuste, performante, schaalbare, redundante, ease of use en kostenefficiënte oplossing te bieden voor de managementtool `Hank`. Om deze doelstellingen te kunnen bereiken is het van uiterst belang om de juiste technologieën te selecteren. Tijdens het onderzoek zal er ook een proof of concept uitgewerkt worden met de vastgestelde technologieën.

%---------- Stand van zaken -> literatuurstudie ------------

\section{State-of-the-art}%
\label{sec:state-of-the-art}

Een framework is een set van tools gebruikt als basis om goed gestructureerde en betrouwbare software en systemen te ontwikkelen. Een web development framework wordt vaak gebruikt bij het ontwikkelen van een web applicatie. Inclusief web services, web resources en web API's.

\subsection{Front-end frameworks}%
\label{sub:frontend_frameworks}
Front-end van een applicatie, vaak duidt dit naar de programmeertalen HTML, CSS en Ja\-va\-script. Deze programmeertalen vormen de basis van een webapplicatie. HTML zorgt voor de structuur van de webpagina, CSS zorgt voor de stijl en Ja\-va\-script zorgt voor de interactie. Maar deze zijn niet voldoende om een volledige webapplicatie te ontwikkelen. Daarom zijn er een aantal frameworks ontwikkeld die de ontwikkeling van een webapplicatie vergemakkelijken. 
Voorbeelden zijn Angular, React, Ambular, Ember, Bootstrap, Vue, \ldots\autocite{Jaiswal2022} 

\subsection{Back-end frameworks}%
\label{sub:backend_frameworks}
Back-end van een applicatie, vaak duidt dit naar de programmeertalen Java, Python, C\#, PHP, Ja\-va\-script\ldots  Deze programmeertalen zorgen voor de logica van de applicatie. Maar deze zijn niet voldoende om een volledige webapplicatie te ontwikkelen. Ook hier zijn er een aantal frameworks ontwikkeld die de ontwikkeling van een webapplicatie vergemakkelijken.
Voorbeelden zijn Spring, Django,.NET, Node.js, \ldots\autocite{Kaluza2019}

\subsection{Push-Based vs. Pull-Based}%
Een belangrijk onderscheid in web development architecturen is het verschil tussen ``Push-Based'' en ``Pull-Based'' architecturen, dat verwijst naar de rol van de server in relatie tot de view layer.\autocite{Lomas2022}
\bigbreak
In ``Push-Based'' architecture start de server met het verzenden (Pushen) van gegevens om vervolgens de resultaten op de view te renderen. Veel MVC (Model-view-controller) frameworks zijn gebaseerd op dit type architectuur. Voorbeelden van push-based technologieën zijn; Django, Ruby on Rails, Symfony, Spring MVC, Stripes en codeIgniter.
\bigbreak
Alternatief, bestaan ook ``Pull-Based'' architecturen, deze haalt de view laag resultaten op van meerdere controllers volgens de vereisten. Voorbeelden van deze architecturen zijn Lift, Tapestry, JBoss Seam en Wicket.
\bigbreak
Dit onderscheid is essentieel bij het kiezen van een geschikt framework, omdat het de manier waarop de gegevens worden uitgewisseld beïnvloedt.

\subsection{Ge\-rel\-at\-eerde studies}%
\label{sub:Gerelateerde_studies}
Er zijn reeds verschillende studies uitgevoerd in verband met de pre\-sta\-ties en schaalbaarheid van frameworks\autocite{Daityari2023}.  Deze onderzoeken omvatten ook analyses van de beschikbaarheid van soft\-ware\-ont\-wik\-kel\-aars die vertrouwd zijn met deze frameworks.

\subsection{React vs Vue vs Angular}%
Angular is ontwikkeld door Google, deze onderscheidt zich als een volledig framework dat is gebaseerd op TypeScript. In vergelijking met React, een populair front-end framework ontwikkeld door Facebook, dat zijn basis heeft in JavaScript en voornamelijk dient als een open-source bibliotheek voor het uitbouwen van gebruikersinterfaces. Bij het gebruik van React voor webapplicatieontwikkeling zijn aanvullende packages vereist. Aan de andere kant is Vue een progressief framework voor het ontwikkelen van gebruikersinterfaces, gecreëerd door Evan You.\autocite{EvanYou2024} Deze wordt best gebruikt in minder complexe, kleinere web applicaties en is ook makkelijker om te leren. Angular daarentegen gaat verder dan alleen een framework; het wordt gepresenteerd als een uitgebreid platform voor het bouwen van single-page client-applicaties met behulp van HTML en TypeScript. Angular wordt vaak ook beschouwd als een framework met een steile leercurve. De keuze tussen Angular, React en Vue hangt af van de behoeften, budget en complexiteit van het project.\autocite{Joshi2023}


\subsection{Node vs .NET}%
NodeJS en .NET zijn beide krachtige frameworks met verschillende karakteristieken. De keuze tussen Node.js en .NET hangt af van de vereisten van het project. Node.js kan uitblinken in situaties waar snelle prestaties en schaalbaarheid van cruciaal belang zijn, terwijl .NET de voorkeur kan hebben bij het bouwen van meer complexe enterprise-toepassingen die moeten samenwerken met de Microsoft-ecosfeer.
\autocite{Hutsulyak2023}


\bigbreak
In de verdere verkenning van ge\-rel\-a\-teer\-de studies is het belangrijk te kijken naar specifieke bevindingen, methodologieën en conclusies die door onderzoekers zijn bereikt. Het identificeren van trends, best practices en e\-ven\-tu\-ele uitdagingen die in eerdere studies naar voren zijn gekomen, kan waardevolle inzichten bieden bij het begrijpen van de rol van frameworks in webontwikkeling en het beoordelen van hun geschiktheid voor specifieke projecten.




%---------- Methodologie -----------------------------------
\section{Methodologie}
Deze paper schetst een methodologisch plan om andere softwareontwikkelaars te informeren over geschikte frameworks voor de ontwikkeling van een full-stack managementtool.
% ---------FASE 1 -----------
\subsection*{Fase 1: Literatuurstudie}
\begin{itemize}
\item \textbf{Doelstelling}: Verzamelen van informatie over front-end en back-end frameworks.
\item \textbf{Aanpak}:
\begin{itemize}
\item Zoeken naar specifieke onderwerpen en bronnen met betrekking tot front-end en back-end frameworks, waaronder pre\-sta\-ties, configuratie en schaalbaarheid.
\item Opstellen van een lijst met mogelijke tools en frameworks die relevant zijn voor de ontwikkeling van managementtools.
\item Onderzoeken van de huidige benaderingen en trends op het gebied van front-end en back-end ontwikkeling.
\end{itemize}
\item \textbf{Tijdskader}: 2 weken
\item \textbf{Deliverable}: Een samenvatting van de vakliteratuur en een lijst met tools en frameworks die relevant zijn voor het ontwikkelen van de man\-age\-ment\-tool ``Hank''.
\end{itemize}
\bigbreak
In deze eerste fase wordt een breed scala aan betrouwbare gegevens verzameld in de vorm van artikels, wetenschappelijke websites, in\-for\-ma\-tie\-ve YouTube videos, tutorials, blogs en berichten op forums. Met behulp van deze bronnen kan er een goede basis gelegd worden voor het onderzoek.

% ---------FASE 2 -----------
\subsection*{Fase 2: Requirement-analysis}
\begin{itemize}
\item \textbf{Doelstelling}: Verwerken van de informatie, functional en non-functional requirements opstellen.
\item \textbf{Aanpak}:
\begin{itemize}
\item Beschrijven van de huidige toestand van frameworks (AS IS).
\item Analyseren van de vereisten van de managementtool (TO BE).
\end{itemize}
\item \textbf{Tijdskader}: 2 weken
\item \textbf{Deliverable}: Een lijst met vereisten gestructureerd volgens de MoSCoW-methode.
\end{itemize}
\bigbreak
In de tweede fase wordt er aan de slag gegaan met het verwerken van de opgenomen informatie uit fase \'{e}\'{e}n, om zo de requirements op te stellen, hierbij wordt er gebruik gemaakt van de huidige toestand van frameworks (AS IS) en de vereisten van de managementtool (TO BE).
\begin{itemize}
\item \textbf{Must have}:
\begin{itemize}
\item Overzicht van consultants.
\item Gebruiksvriendelijke Interface met Verschillende Screen Sizes
\end{itemize}

\item \textbf{Should have}:
\begin{itemize}
  \item Grafieken voor Vooruitgangsbepaling van Consultants
  \item Real-time Updates met Notificatie Systeem
  \item Interactieve Gebruikersprofielen


\end{itemize}
\end{itemize} 
\bigskip
% ---------FASE 3 -----------
\subsection*{Fase 3: Ontwerp}
\begin{itemize}
\item \textbf{Doelstelling}: Een werkend ontwerp van de managementtool.
\item \textbf{Aanpak}:
\begin{itemize}
\item Geschikte back-end en front-end kiezen voor het ontwikkelen van de managementtool.
\end{itemize}
\item \textbf{Tijdskader}: 3 weken
\item \textbf{Deliverable}: Een ontwerp voor de oplossing; een short list met de meest geschikte tools en frameworks.
\end{itemize}
\bigbreak
In de derde fase wordt er aan de slag gegaan om de managementapplicatie te ontwikkelen. Hierbij wordt de managementapplicatie ontwikkeld door de gevonden technologieën. Hierbij wordt er gekeken naar de vereisten uit fase 2.
\bigbreak
% ---------FASE 4 -----------
\subsection*{Fase 4: Implementatie}
\begin{itemize}
\item \textbf{Doelstelling}:Implementeren oplossingen, fouten documenteren.
\item \textbf{Aanpak}:
\begin{itemize}
\item Implementeren van de gevonden oplossing.
\item Testen van de oplossing en eventuele problemen oplossen. Analyseren dat systeem voldoet aan de vereisten, en ook nog mogelijke uitbreiding ondersteunt.
\end{itemize}
\item \textbf{Tijdskader}: 4 weken
\item \textbf{Deliverable}: Een uitgebouwde ge\-re\-vi\-seer\-de server met daarop de vereiste software pakketten.
\end{itemize}
\bigbreak
In deze fase wordt de applicatie uitgebouwd. Dit wil zeggen dat ervoor gezorgd wordt dat aan alle software vereisten wordt voldaan. Wanneer dit aspect van de implementatie fase af is, kan er verder op deze applicatie gewerkt worden. Hier bovenop worden de benodigde softwarepakketten geïnstalleerd zodat de server voldoet aan de vereisten uit fase 2.
\bigskip
% ---------FASE 5 -----------
\subsection*{Fase 5: Evaluatie}
\begin{itemize}
\item \textbf{Doelstelling}: Opstellen conclusie na revisie docenten.
\item \textbf{Aanpak}:
\begin{itemize}
\item Oplossing evalueren.
\item Vergelijken van de oplossing met andere oplossingen.
\item Opstellen van een conclusie.
\end{itemize}
\item \textbf{Tijdskader}: 2 weken
\item \textbf{Deliverable}: Een evaluatie van de operationele applicatie en deze staven aan de vereisten. Hieruit kan er een conclusie getrokken.
\end{itemize}


%---------- Verwachte resultaten ----------------------------------------------
\pagebreak
\section{Verwacht resultaat en conclusie}%
\label{sec:verwachte_resultaten}
Het verwachte resultaat van het onderzoek is een diepgaand inzicht in de pre\-sta\-ties, schaalbaarheid en gebruiksvriendelijkheid van de getoetste frameworks, met een specifieke focus op het gebruik van dotnet als backend en React als frontend. Hier zijn enkele mogelijke verwachtingen en hypotheses die kunnen worden geformuleerd:

\begin{itemize}
  \item \textbf{Prestaties:}:
    \begin{itemize}
      \item \textbf{Backend (dotnet)}: Er wordt verwacht dat dotnet goede prestaties zal leveren, gezien de optimalisaties en efficiënties die inherent zijn aan het framework.
      \item \textbf{Frontend (React)}: React staat bekend om zijn snelle weergave en efficiënte virtual DOM.  Hoge reactiesnelheid en vloeiende gebruikersinteractie worden verwacht.
    \end{itemize}
  \item \textbf{Betrouwbaarheid:}
    \begin{itemize}
      \item \textbf{Backend (dotnet)}: Er wordt verwacht dat dotnet, met zijn ondersteuning van microsoft een betrouwbare keuze zal zijn
      \item \textbf{Frontend (React)}: React's frequente updates wijzen op een betrouwbare en actieve community, wat een positief teken is voor de betrouwbaarheid van het framework.
    \end{itemize}
  \item \textbf{Schaalbaarheid:}
    \begin{itemize}
      \item \textbf{Backend (dotnet)}: Dotnet zou moeten schalen bij toenemende werkbelasting, waarbij de architectuur en ingebouwde tools helpen om het systeem soepel te laten groeien.
      \item \textbf{Frontend (React)}: React staat bekend om zijn snelle weergave en efficiënte virtual DOM.Hoge reactiesnelheid en vloeiende gebruikersinteractie worden verwacht.
    \end{itemize}
  \item \textbf{Gebruiksvriendelijkheid:}
    \begin{itemize}
      \item \textbf{Backend (dotnet)}: Er wordt verwacht dat dotnet, met zijn uitgebreide ontwikkelingsomgeving en ondersteunende tools, een gebruiksvriendelijke backend-ervaring biedt.
      \item \textbf{Frontend (React)}: React's declaratieve aanpak en herbruikbare componenten zouden moeten resulteren in een efficiënte en gemakkelijk te onderhouden frontend.
    \end{itemize}
\end{itemize}

\subsection{Conclusie}
Op basis van de verwachte resultaten kan een conclusie worden getrokken over welke combinatie van frameworks (dotnet als backend en React als frontend) het meest geschikt is voor specifieke softwareontwikkelingsbehoeften. Deze conclusie kan dienen als waardevolle informatie voor softwareontwikkelaars en besluitvormers bij het kiezen van technologie stacks voor toekomstige projecten.

De meerwaarde voor de doelgroep is het verkrijgen van concrete inzichten en aanbevelingen op basis van objectieve metingen en evaluaties. Hierdoor kunnen ontwikkelaars van weloverwogen beslissingen nemen bij het selecteren van frameworks, rekening houdend met pre\-sta\-ties, schaalbaarheid en gebruiksvriendelijkheid. Het uiteindelijke resultaat van de bachelorproef biedt praktische richtlijnen die de efficiëntie en effectiviteit van softwareontwikkeling kunnen verbeteren.
\printbibliography[heading=bibintoc]

\clearpage
\begin{ganttchart}[
  x unit=0.6cm,
  y unit title=0.6cm,
  y unit chart=0.8cm,
  vgrid,hgrid,
  title label anchor/.style={below=-1.6ex},
  title left shift=.05,
  title right shift=-.05,
  title height=1,
  progress label text={},
  bar height=0.5,
  group right shift=0,
  group top shift=.6,
  group height=.3,
  group peaks tip position=0
  ]{1}{16}
%labels
\gantttitle{2024}{16} \\
\gantttitle{Feb}{4}
\gantttitle{Mar}{4}
\gantttitle{Apr}{4}
\gantttitle{May}{4}\\

%tasks
\ganttgroup{Fase 1: Literatuurstudie}{1}{2} \\
\ganttbar{Informatie opzoeken}{1}{2} \\
\ganttbar{Vergelijken en analyseren}{2}{3} \\
\ganttmilestone{Fase 1 Afgerond}{3} \\

\ganttgroup{Fase 2: Requirement-analysis}{4}{6} \\
\ganttbar{Beschrijven AS IS}{4}{5} \\
\ganttbar{Analyseren TO BE}{5}{6} \\
\ganttmilestone{Fase 2 Afgerond}{6} \\

\ganttgroup{Fase 3: Ontwerp}{7}{9} \\
\ganttbar{Back-end Frameworks zoeken}{7}{8} \\
\ganttbar{Front-end Frameworks zoeken}{7}{8} \\
\ganttbar{Applicatie Ontwerpen}{8}{9} \\
\ganttmilestone{Fase 3 Afgerond}{9} \\

\ganttgroup{Fase 4: Implementatie}{9}{13} \\
\ganttbar{Applicatie Ontwikkelen}{11}{12} \\
\ganttbar{Implementatie oplossingen}{12}{13} \\
\ganttmilestone{Fase 4 Afgerond}{13} \\

\ganttgroup{Fase 5: Evaluatie}{13}{15} \\
\ganttbar{Oplossing evalueren}{13}{14} \\
\ganttbar{Vergelijken met andere oplossingen}{13}{14} \\
\ganttbar{Conclusie opstellen}{14}{15} \\
\ganttmilestone{Fase 5 Afgerond}{15} \\

\ganttmilestone{Eindoplevering}{16} \\

%relations
\ganttlink{elem2}{elem3}
\ganttlink{elem5}{elem6}
\ganttlink{elem8}{elem9}
\ganttlink{elem11}{elem12}
\ganttlink{elem14}{elem15}
\ganttlink{elem17}{elem18}
\ganttlink{elem20}{elem21}
\end{ganttchart}
\end{document}