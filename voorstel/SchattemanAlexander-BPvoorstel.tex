%==============================================================================
% Sjabloon onderzoeksvoorstel bachproef
%==============================================================================
% Gebaseerd op document class `hogent-article'
% zie <https://github.com/HoGentTIN/latex-hogent-article>

% Voor een voorstel in het Engels: voeg de documentclass-optie [english] toe.
% Let op: kan enkel na toestemming van de bachelorproefcoördinator!
\documentclass{hogent-article}

% Invoegen bibliografiebestand
\addbibresource{voorstel.bib}

% Informatie over de opleiding, het vak en soort opdracht
\studyprogramme{Professionele bachelor toegepaste informatica}
\course{Bachelorproef}
\assignmenttype{Onderzoeksvoorstel}
% Voor een voorstel in het Engels, haal de volgende 3 regels uit commentaar
% \studyprogramme{Bachelor of applied information technology}
% \course{Bachelor thesis}
% \assignmenttype{Research proposal}

\academicyear{2023-2024} % TODO: pas het academiejaar aan

% TODO: Werktitel
\title{Optimalisatie van Frontend- en Backend-Technologieën voor de Ontwikkeling van de Managementtool "Hank"}

% TODO: Studentnaam en emailadres invullen
\author{Alexander Schatteman}
\email{Alexander.schatteman@student.hogent.be}

% TODO: Geef de co-promotor op
\supervisor[Co-promotor]{K. DeRudder (Elmos NV, \href{mailto:kevin.derudder@elmos.be}{kevin.derudder@elmos.be})}

% Binnen welke specialisatierichting uit 3TI situeert dit onderzoek zich?
% Kies uit deze lijst:
%
% - Mobile \& Enterprise development
% - AI \& Data Engineering
% - Functional \& Business Analysis
% - System \& Network Administrator
% - Mainframe Expert
% - Als het onderzoek niet past binnen een van deze domeinen specifieer je deze
%   zelf
%
\specialisation{Mobile \& Enterprise development}
\keywords{Scheme, World Wide Web, $\lambda$-calculus}

\begin{document}
Deze paper gaat over het analyseren welke frontend en backend Technologieën het best 
gebruikt kunnen worden bij het ontwikkelen van een managementtool "Hank". 
De term frontend verwijst naar de gebruikersinterface van een website of applicatie.
De term backend verwijst naar de server, applicatie en database van een website of applicatie.
De managementtool zal gebruikt worden om de verschillende processen binnen het bedrijf te beheren. 
De frontend zal gebruikt worden om de gebruiker een overzicht te geven van de verschillende processen en 
de backend zal gebruikt worden om de data te verwerken en op te slaan.
Deze paper zal een vergelijking maken tussen de verschillende frontend en backend technologieën. 
De vergelijking zal gebeuren op basis van de performantie, de veiligheid en de kostprijs van de technologieën.
\begin{abstract}


\end{abstract}

\tableofcontents

% De hoofdtekst van het voorstel zit in een apart bestand, zodat het makkelijk
% kan opgenomen worden in de bijlagen van de bachelorproef zelf.
\input{voorstel-inhoud}

\printbibliography[heading=bibintoc]

\end{document}