%==============================================================================
% Sjabloon onderzoeksvoorstel bachproef
%==============================================================================
% Gebaseerd op document class `hogent-article'
% zie <https://github.com/HoGentTIN/latex-hogent-article>
%
\documentclass{hogent-article}
\usepackage{pgfgantt}
\usepackage{tikz}
% Invoegen bibliografiebestand
\addbibresource{voorstel.bib}

% Informatie over de opleiding, het vak en soort opdracht
\studyprogramme{Professionele bachelor toegepaste informatica}
\course{Bachelorproef}
\assignmenttype{Onderzoeksvoorstel}

\academicyear{2023-2024}

% TODO: Werktitel
\title{Onderzoek naar het best geschikte front- en backend technologieën voor de ontwikkeling van de managementtool `Hank`}

\author{Alexander Schatteman}
\email{Alexander.schatteman@student.hogent.be}

% TODO: Geef de co-promotor op
\supervisor[Co-promotor]{K. DeRudder (Elmos NV, \href{mailto:kevin.derudder@elmos.be}{kevin.derudder@elmos.be})}

\projectrepo{https://github.com/LexDarcoz/HoGent_BP_SchattemanAlexander}
\specialisation{Mobile \& Enterprise development}
\keywords{Frontend, Backend, Interface, Managementtool, HANK, Frameworks, Kostprijs, Performantie, Betrouwbaarheid, Schaalbaarheid}

% ----------- Begin Paper --------------
\begin{document}

\begin{abstract}
    In een snel evoluerende wereld van softwareontwikkeling en innovatie is het cruciaal om de meest geschikte technologieën te selecteren bij het ontwikkelen van een applicatie of tool die het best voldoet aan de specifieke behoeften van de klant.
    \bigskip
    Het hoofddoel van deze paper is het onderzoeken welke combinatie van front- en backend technologieën het meest geschikt is voor de ontwikkeling van de managementtool `Hank`.
    \bigskip
    Om de context van `HANK' te verduidelijken: `HANK' is een managementtool dat tijdens dit onderzoek ook daadwerkelijk zal ontwikkeld worden. In opdracht van consultancy bedrijf Elmos NV, heeft `HANK' tot doel de bedrijfsprocessen van het Human Resources-departement efficiënter te maken door middel van een op maat gemaakte managementtool. Deze webapplicatie zou probleemloos hand in hand moeten gaan met een efficiente front- en backend technologie. 
    \bigskip
    De vergelijking zal worden gemaakt op basis van performantie, betrouwbaarheid en schaalbaarheid. In context van dit onderzoek verwijst de term `Frontend-technologieën' naar de technologische componenten die worden ingezet om de gebruiker een intuïtieve interface te bieden voor het beheren van diverse bedrijfsprocessen. Deze interface communiceert met de backend technologie die de data verwerkt en opslaat. 
    \bigskip
    Dit onderzoek beoogt een beter inzicht te bezorgen in de verschillende technologieën en zo een kostefficiente en performante oplossing te kunnen bieden voor de managementtool `Hank'.
\end{abstract}

\tableofcontents
%---------- Inleiding ----------------------
\section{Introductie}%
\label{sec:introductie}
  In een snel evoluerende wereld van softwareontwikkeling is het cruciaal om zorgvuldig de meest geschikte technologieën te selecteren bij het ontwikkelen van een applicatie of tool die het best voldoet aan de specifieke behoeften van de klant.
  Het kiezen van de juiste technologieën is essentieel, aangezien ze de ruggengraat van het eindresultaat vormen en bepalen het succes ervan. Dit is van uiterst belang om zo te kunnen voldoen aan de specifieke bijnodigdheden en behoeften van de klant, waardoor tijdsverlies en onnodige kosten zo goed mogelijk worden vermeden. 
  
  \subsection{Probleemstelling}%
  \label{sub:probleemstelling}
  Dit onderzoek richt zich op de ontwikkeling van de managementtool `Hank', met als doel een diepgaande analyse te maken van de verschillende front- en backend-technologieën. Hank: een Elmos-inhouse visualisatie-tool die de Competence Centers in een live dashboard stopt. Elke collega/consultant heeft een groeipad. Hank toont hiervan de live leertrajecten, certificaten maar ook examens en geplande opleidingen. `Hank' is een webapplicatie die in opdracht van het consultancy bedrijf Elmos NV ontwikkeld zal worden. Hetgeen dat Elmos wil bereiken is het versnellen van bedrijfsprocessen door middel van een op maat gemaakte managementapplicatie. Deze webapplicatie zou probleemloos hand in hand gaan met een efficiente front- en backend technologie.
  \bigskip
  Als er over backend programmeertalen gesproken wordt, zijn een aantal populaire opties zoals Java, Python, C\#, PHP, Javascript, \ldots Hierop zijn er een aantal frameworks gebaseerd zoals Spring, Django,.NET, Laravel, Node.js, \ldots Deze frameworks zijn uniek in hun eigen manier en hebben elk hun eigen voor- en nadelen. Bij frontend technologieën zijn er ook een aantal populaire opties zoals Javascript, Typescript, HTML, CSS, \ldots Hierop zijn respectievelijk ook een aantal frameworks gebaseerd zoals Angular, React, Vue, \ldots.
  \bigskip
  Er wordt gestreeft naar een robuuste, performante, schaalbare, redundante, ease of use en kostefficiënte oplossing te bieden voor de managementtool `Hank`. Om deze doelstellingen te kunnen bereiken is het van uiterst belang om de juiste technologieën te selecteren. Tijdens ons onderzoek zal er ook een proof of concept worden uitgewerkt worden met de vastgestelde technologieën.

%---------- Stand van zaken -> literatuurstudie ------------

\section{State-of-the-art}%
\label{sec:state-of-the-art}

Een framework is een set van tools gebruikt als basis om goed gestructureerde en betrouwbare software en systemen te ontwikkelen. Een web development framework wordt vaak gebruikt bij het ontwikkelen van een web applicatie. Inclusief web services, web resources en web API's.

\subsection{Front-end frameworks}%
\label{sub:frontend_frameworks}
Front-end van een applicatie, vaak duidt dit naar de programmeertalen HTML, CSS en Javascript. Deze programmeertalen vormen de basis van een webapplicatie. HTML zorgt voor de structuur van de webpagina, CSS zorgt voor de stijl en Javascript zorgt voor de interactie. Maar deze zijn niet voldoende om een volledige webapplicatie te ontwikkelen. Daarom zijn er een aantal frameworks ontwikkeld die de ontwikkeling van een webapplicatie vergemakkelijken. 
Voorbeelden van front-end frameworks zijn Angular, React, Ambular, Ember, Bootstrap, Vue, \ldots\textcite{Jaiswal2022} 

\subsection{Back-end frameworks}%
\label{sub:backend_frameworks}
Back-end van een applicatie, vaak duidt dit naar de programmeertalen Java, Python, C\#, PHP, Javascript, \ldots Deze programmeertalen zorgen voor de logica van de applicatie. Maar deze zijn niet voldoende om een volledige webapplicatie te ontwikkelen. Ook hier zijn er een aantal frameworks ontwikkeld die de ontwikkeling van een webapplicatie vergemakkelijken.
Voorbeelden van back-end frameworks zijn Spring, Django,.NET, Laravel, Node.js, \ldots\textcite{Kaluza2019}

\subsection{Push-Based vs Pull-Based}%
Een belangrijk onderscheid in web development architecturen is het verschil tussen ``Push-Based'' en ``Pull-Based'' architecturen, dat verwijst naar de rol van de server in relatie tot de view layer.\textcite{Lomas2022}
\bigskip
In ``Push-Based'' architecture start de server met het verzenden (Pushen) van gegevens om vervolgens de resultaten op de view te renderen. Veel MVC (Model-view-controller) frameworks zijn gebaseerd op dit type architectuur. Voorbeelden van push-based technologieën zijn; Django, Ruby on Rails, Symfony, Spring MVC, Stripes en codeIgniter.
\bigskip
Alternatief, heb je ook ``Pull-Based'' architecturen, deze haalt de view laag resulaten op van meerdere controllers volgens de vereisten. Voorbeelden van deze architecturen zijn Lift, Tapestry, JBoss Seam en Wicket.
\bigskip
Dit onderscheid is esentieel bij het kiezen van een geschikt framework, omdat het de manier waarop de gegevens worden uitgewisseld beïnvloedt.
\subsection{Ge\-rel\-at\-eerde studies}%
\label{sub:Gerelateerde_studies}
Er zijn reeds verschillende studies uitgevoerd in verband met de prestaties en schaalbaarheid van frameworks. Deze onderzoeken omvatten ook analyses van de beschikbaarheid van soft\-ware\-ont\-wik\-kel\-aars die vertrouwd zijn met deze frameworks.
\bigskip
In de verdere verkenning van ge\-rel\-a\-teer\-de studies is het belangrijk te kijken naar specifieke bevindingen, methodologieën en conclusies die door onderzoekers zijn bereikt. Het identificeren van trends, best practices en e\-ven\-tu\-ele uitdagingen die in eerdere studies naar voren zijn gekomen, kan waardevolle inzichten bieden bij het begrijpen van de rol van frameworks in webontwikkeling en het beoordelen van hun geschiktheid voor specifieke projecten.
\pagebreak
%---------- Methodologie -----------------------------------
\subsection{Methodologie}
Deze paper schetst een methodologisch plan om andere softwareontwikkelaars te informeren over geschikte frameworks voor de ontwikkeling van een full-stack managementtool.
% ---------FASE 1 -----------
\subsection*{Fase 1: Literatuurstudie}
\begin{itemize}
\item \textbf{Doelstelling}: Verzamelen van informatie over front-end en back-end frameworks.
\item \textbf{Aanpak}:
\begin{itemize}
\item Zoeken naar specifieke onderwerpen en bronnen met betrekking tot front-end en back-end frameworks, waaronder prestaties, configuratie en schaalbaarheid.
\item Opstellen van een lijst met mogelijke tools en frameworks die relevant zijn voor de ontwikkeling van managementtools.
\item Onderzoeken van de huidige benaderingen en trends op het gebied van front-end en back-end ontwikkeling.
\end{itemize}
\item \textbf{Tijdskader}: 2 weken
\item \textbf{Deliverable}: Een samenvatting van de vakliteratuur en een lijst met tools en frameworks die relevant zijn voor het ontwikkelen van de man\-age\-ment\-tool ``Hank''.
\end{itemize}
\bigbreak
In deze eerste fase wordt een breed scala aan betrouwbare gegevens verzameld in de vorm van artikels, wetenschappelijke websites, informatieve YouTube videos, tutorials, blogs en berichten op forums. Met behulp van deze bronnen kan er een goede basis gelegd worden voor het onderzoek.

% ---------FASE 2 -----------
\subsection*{Fase 2: Requirement-analysis}
\begin{itemize}
\item \textbf{Doelstelling}: Verwerken van de informatie, functional en non-functional requirements opstellen.
\item \textbf{Aanpak}:
\begin{itemize}
\item Beschrijven van de huidige toestand van frameworks (AS IS).
\item Analyseren van de vereisten van de managementtool (TO BE).
\end{itemize}
\item \textbf{Tijdskader}: 2 weken
\item \textbf{Deliverable}: Een lijst met vereisten gestructureerd volgens de MoSCoW-methode.
\end{itemize}
\bigskip
In de tweede fase wordt er aan de slag gegaan met het verwerken van de opgenomen informatie uit fase \'{e}\'{e}n, om zo de requirements op te stellen, hierbij wordt er gebruik gemaakt van de huidige toestand van frameworks (AS IS) en de vereisten van de managementtool (TO BE).
\begin{itemize}
\item \textbf{Must have}:
\begin{itemize}
\item TODO
\end{itemize}

\item \textbf{Should have}:
\begin{itemize}
  \item TODO
  \item TODO
  \item TODO

\end{itemize}
\end{itemize}
\bigskip
% ---------FASE 3 -----------
\subsection*{Fase 3: Ontwerp}
\begin{itemize}
\item \textbf{Doelstelling}: Een werkend ontwerp van de managementtool.
\item \textbf{Aanpak}:
\begin{itemize}
\item Geschikte back-end en front-end kiezen voor het ontwikkelen van de managementtool.
\end{itemize}
\item \textbf{Tijdskader}: 3 weken
\item \textbf{Deliverable}: Een ontwerp voor de oplossing; een short list met de meest geschikte tools en frameworks.
\end{itemize}
\bigskip
In de derde fase wordt er aan de slag gegaan om de managementapplicatie te ontwikkelen. Hierbij wordt de managementapplicatie ontwikkeld door de gevonden technologieën. Hierbij wordt er gekeken naar de vereisten uit fase 2.
\bigbreak
% ---------FASE 4 -----------
\subsection*{Fase 4: Implementatie}
\begin{itemize}
\item \textbf{Doelstelling}:Implementeren oplossingen, fouten documenteren.
\item \textbf{Aanpak}:
\begin{itemize}
\item Implementeren van de gevonden oplossing.
\item Testen van de oplossing en eventuele problemen oplossen. Analyseren dat systeem voldoet aan de vereisten, en ook nog mogelijke uitbreiding ondersteunt.
\end{itemize}
\item \textbf{Tijdskader}: 4 weken
\item \textbf{Deliverable}: Een uitgebouwde ge\-re\-vi\-seer\-de server met daarop de vereiste software pakketten.
\end{itemize}
\bigbreak
In deze fase wordt de applicatie uitgebouwd. Dit wil zeggen dat ervoor gezorgd wordt dat aan alle software vereisten wordt voldaan. Wanneer dit aspect van de implementatie fase af is, kan er verder op deze applicatie gewerkt worden. Hier bovenop worden de benodigde softwarepakketten geïnstalleerd zodat de server voldoet aan de vereisten uit fase 2.
\bigskip
% ---------FASE 5 -----------
\subsection*{Fase 5: Evaluatie}
\begin{itemize}
\item \textbf{Doelstelling}: Opstellen conclusie na revisie docenten.
\item \textbf{Aanpak}:
\begin{itemize}
\item Oplossing evalueren.
\item Vergelijken van de oplossing met andere oplossingen.
\item Opstellen van een conclusie.
\end{itemize}
\item \textbf{Tijdskader}: 2 weken
\item \textbf{Deliverable}: Een evaluatie van de operationele applicatie en deze staven aan de vereisten. Hieruit kan er een conclusie getrokken.
\end{itemize}


%---------- Verwachte resultaten ----------------------------------------------
\section{Verwacht resultaat en conclusie}%
\label{sec:verwachte_resultaten}




Hier beschrijf je welke resultaten je verwacht. Als je metingen en simulaties uitvoert, kan je hier al mock-ups maken van de grafieken samen met de verwachte conclusies. Benoem zeker al je assen en de onderdelen van de grafiek die je gaat gebruiken. Dit zorgt ervoor dat je concreet weet welk soort data je moet verzamelen en hoe je die moet meten.

Wat heeft de doelgroep van je onderzoek aan het resultaat? Op welke manier zorgt jouw bachelorproef voor een meerwaarde?

Hier beschrijf je wat je verwacht uit je onderzoek, met de motivatie waarom. Het is \textbf{niet} erg indien uit je onderzoek andere resultaten en conclusies vloeien dan dat je hier beschrijft: het is dan juist interessant om te onderzoeken waarom jouw hypothesen niet overeenkomen met de resultaten.

\printbibliography[heading=bibintoc]

\clearpage
\begin{ganttchart}[
  x unit=0.45cm,
  y unit title=0.6cm,
  y unit chart=0.8cm,
  vgrid,hgrid,
  title label anchor/.style={below=-1.6ex},
  title left shift=.05,
  title right shift=-.05,
  title height=1,
  progress label text={},
  bar height=0.5,
  group right shift=0,
  group top shift=.6,
  group height=.3,
  group peaks tip position=0
  ]{1}{24}
%labels
\gantttitle{2024}{24} \\
\gantttitle{Jan}{4}
\gantttitle{Feb}{4}
\gantttitle{Mar}{4}
\gantttitle{Apr}{4}
\gantttitle{May}{4}
\gantttitle{Jun}{4} \\

%tasks
\ganttgroup{Fase 1: Literatuurstudie}{1}{3} \\
\ganttbar{Informatie opzoeken}{1}{2} \\
\ganttbar{Vergelijken en analyseren}{2}{3} \\
\ganttmilestone{Fase 1 Afgerond}{3} \\

\ganttgroup{Fase 2: Requirement-analysis}{4}{6} \\
\ganttbar{Beschrijven AS IS}{4}{5} \\
\ganttbar{Analyseren TO BE}{5}{6} \\
\ganttmilestone{Fase 2 Afgerond}{6} \\

\ganttgroup{Fase 3: Ontwerp}{7}{10} \\
\ganttbar{Back-end Frameworks zoeken}{7}{8} \\
\ganttbar{Front-end Frameworks zoeken}{8}{9} \\
\ganttbar{Applicatie Ontwerpen}{9}{10} \\
\ganttmilestone{Fase 3 Afgerond}{10} \\

\ganttgroup{Fase 4: Implementatie}{11}{16} \\
\ganttbar{Applicatie Ontwikkelen}{11}{12} \\
\ganttbar{Implementatie oplossingen}{13}{14} \\
\ganttmilestone{Fase 4 Afgerond}{16} \\

\ganttgroup{Fase 5: Evaluatie}{17}{18} \\
\ganttbar{Oplossing evalueren}{17}{18} \\
\ganttbar{Vergelijken met andere oplossingen}{18}{19} \\
\ganttbar{Conclusie opstellen}{19}{20} \\
\ganttmilestone{Fase 5 Afgerond}{20} \\

\ganttmilestone{Eindoplevering}{22} \\

%relations
\ganttlink{elem2}{elem3}
\ganttlink{elem5}{elem6}
\ganttlink{elem8}{elem9}
\ganttlink{elem11}{elem12}
\ganttlink{elem14}{elem15}
\ganttlink{elem17}{elem18}
\ganttlink{elem20}{elem21}
\end{ganttchart}
\end{document}